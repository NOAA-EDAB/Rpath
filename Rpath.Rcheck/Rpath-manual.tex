\nonstopmode{}
\documentclass[a4paper]{book}
\usepackage[times,inconsolata,hyper]{Rd}
\usepackage{makeidx}
\usepackage[utf8,latin1]{inputenc}
% \usepackage{graphicx} % @USE GRAPHICX@
\makeindex{}
\begin{document}
\chapter*{}
\begin{center}
{\textbf{\huge Package `Rpath'}}
\par\bigskip{\large \today}
\end{center}
\begin{description}
\raggedright{}
\item[Type]\AsIs{Package}
\item[Title]\AsIs{R implementation of Ecopath with Ecosim}
\item[Version]\AsIs{0.0.0.9000}
\item[Date]\AsIs{2015-02-09}
\item[Description]\AsIs{This package implements the core equations from the popular
Ecopath with Ecosim mass balance model. (Need to expand)}
\item[License]\AsIs{GPL (>= 2)}
\item[Imports]\AsIs{data.table, MASS, Rcpp (>= 0.11.3)}
\item[LinkingTo]\AsIs{Rcpp}
\item[NeedsCompilation]\AsIs{yes}
\item[Author]\AsIs{Kerim Aydin [aut],
Sean Lucey [aut, cre],
Sarah Gaichas [aut]}
\item[Maintainer]\AsIs{Sean Lucey }\email{Sean.Lucey@NOAA.gov}\AsIs{}
\item[Archs]\AsIs{i386, x64}
\end{description}
\Rdcontents{\R{} topics documented:}
\inputencoding{utf8}
\HeaderA{Rpath-package}{What the package does (short line)}{Rpath.Rdash.package}
\aliasA{Rpath}{Rpath-package}{Rpath}
\keyword{package}{Rpath-package}
%
\begin{Description}\relax
More about what it does (maybe more than one line)
\textasciitilde{}\textasciitilde{} A concise (1-5 lines) description of the package \textasciitilde{}\textasciitilde{}
\end{Description}
%
\begin{Details}\relax

\Tabular{ll}{
Package: & Rpath\\{}
Type: & Package\\{}
Version: & 1.0\\{}
Date: & 2015-02-09\\{}
License: & GPL (>= 2)\\{}
}
\textasciitilde{}\textasciitilde{} An overview of how to use the package, including the most important functions \textasciitilde{}\textasciitilde{}
\end{Details}
%
\begin{Author}\relax
Your Name

Maintainer: Your Name <your@email.com>
\end{Author}
%
\begin{References}\relax
\textasciitilde{}\textasciitilde{} Literature or other references for background information \textasciitilde{}\textasciitilde{}
\end{References}
%
\begin{SeeAlso}\relax
\textasciitilde{}\textasciitilde{} Optional links to other man pages, e.g. \textasciitilde{}\textasciitilde{}
\textasciitilde{}\textasciitilde{} \code{\LinkA{<pkg>}{<pkg>}} \textasciitilde{}\textasciitilde{}
\end{SeeAlso}
\inputencoding{utf8}
\HeaderA{check.rpath.param}{Check Rpath parameter files}{check.rpath.param}
%
\begin{Description}\relax
Logical check that the parameter files are filled out correctly, i.e. data is entered where it is
expected.
\end{Description}
%
\begin{Usage}
\begin{verbatim}
check.rpath.param(filename, parameter = "model")
\end{verbatim}
\end{Usage}
%
\begin{Arguments}
\begin{ldescription}
\item[\code{filename}] Name of the parameter file.  Must be a .csv.

\item[\code{parameter}] The type of parameter file you are checking.  Choices include "model",
"diet", "juvenile", and "pedigree".
\end{ldescription}
\end{Arguments}
%
\begin{Value}
Checks Rpath parameter files for consistency.  An error message will be produced if one of
the logical checks fails.  Checks include:
(NOTE: This does not ensure data is correct just that it is in the right places).
\end{Value}
%
\begin{SeeAlso}\relax
Other Rpath.functions: \code{\LinkA{create.rpath.param}{create.rpath.param}};
\code{\LinkA{ecopath}{ecopath}}; \code{\LinkA{ecosim.init}{ecosim.init}};
\code{\LinkA{ecosim.plot}{ecosim.plot}}; \code{\LinkA{ecosim.run}{ecosim.run}};
\code{\LinkA{webplot}{webplot}}; \code{\LinkA{write.Rpath.sim}{write.Rpath.sim}};
\code{\LinkA{write.Rpath}{write.Rpath}}
\end{SeeAlso}
\inputencoding{utf8}
\HeaderA{create.rpath.param}{Create shells for the Rpath parameter files}{create.rpath.param}
%
\begin{Description}\relax
Creates a shell of the parameter files that can then be filled out in R, Excel, or
another spreadsheet program.
\end{Description}
%
\begin{Usage}
\begin{verbatim}
create.rpath.param(filename = NA, group, type = NA, parameter = "model")
\end{verbatim}
\end{Usage}
%
\begin{Arguments}
\begin{ldescription}
\item[\code{filename}] Name of the output file saved as a .csv. If NA the file will not be written.

\item[\code{group}] Vector of group names.  If parameter equals "juvenile", this should be a vector of
stanza groups only (Juvenile and Adults in one group).

\item[\code{type}] Numeric vector of group type. Living = 0, Producer = 1, Detritus = 2,
Fleet = 3. Default NA is used for the juvenile and pedigree parameter files.

\item[\code{parameter}] The type of parameter file you are creating.  Choices include "model",
"diet", "juvenile", and "pedigree".
\end{ldescription}
\end{Arguments}
%
\begin{Value}
Outputs a shell of the parameter file indicated by the parameter variable.  The shell
is populated with values of NA or logical default values.  Values can then be filled in using
R or any spreadsheet program.  Use check.rpath.param() to ensure parameter files are filled out
correctly (NOTE: This does not ensure data is correct just that it is in the right places).
\end{Value}
%
\begin{SeeAlso}\relax
Other Rpath.functions: \code{\LinkA{check.rpath.param}{check.rpath.param}};
\code{\LinkA{ecopath}{ecopath}}; \code{\LinkA{ecosim.init}{ecosim.init}};
\code{\LinkA{ecosim.plot}{ecosim.plot}}; \code{\LinkA{ecosim.run}{ecosim.run}};
\code{\LinkA{webplot}{webplot}}; \code{\LinkA{write.Rpath.sim}{write.Rpath.sim}};
\code{\LinkA{write.Rpath}{write.Rpath}}
\end{SeeAlso}
\inputencoding{utf8}
\HeaderA{ecopath}{Ecopath modual of Rpath}{ecopath}
%
\begin{Description}\relax
Performs initial mass balance using a model parameter file and diet
matrix file.
\end{Description}
%
\begin{Usage}
\begin{verbatim}
ecopath(modfile, dietfile, pedfile, eco.name = NA)
\end{verbatim}
\end{Usage}
%
\begin{Arguments}
\begin{ldescription}
\item[\code{modfile}] Comma deliminated model parameter file.

\item[\code{dietfile}] Comma deliminated diet matrix file.

\item[\code{pedfile}] Comma deliminated pedigree file.

\item[\code{eco.name}] Optional name of the ecosystem which becomes an attribute of
rpath object.
\end{ldescription}
\end{Arguments}
%
\begin{Value}
Returns an Rpath object that can be supplied to the ecosim.init function.
\end{Value}
%
\begin{SeeAlso}\relax
Other Rpath.functions: \code{\LinkA{check.rpath.param}{check.rpath.param}};
\code{\LinkA{create.rpath.param}{create.rpath.param}};
\code{\LinkA{ecosim.init}{ecosim.init}}; \code{\LinkA{ecosim.plot}{ecosim.plot}};
\code{\LinkA{ecosim.run}{ecosim.run}}; \code{\LinkA{webplot}{webplot}};
\code{\LinkA{write.Rpath.sim}{write.Rpath.sim}}; \code{\LinkA{write.Rpath}{write.Rpath}}
\end{SeeAlso}
\inputencoding{utf8}
\HeaderA{ecosim.init}{Initial set up for Ecosim modual of Rpath}{ecosim.init}
%
\begin{Description}\relax
Performs initial set up for Ecosim by converting ecopath values to rates,
initializing stanzas, and packing everything together to run.
\end{Description}
%
\begin{Usage}
\begin{verbatim}
ecosim.init(Rpath, juvfile, YEARS = 100)
\end{verbatim}
\end{Usage}
%
\begin{Arguments}
\begin{ldescription}
\item[\code{Rpath}] Rpath object containing a balanced model.

\item[\code{juvfile}] Comma deliminated file with multi-stanza parameters.

\item[\code{YEARS}] Integer value to set maximum number of years.
\end{ldescription}
\end{Arguments}
%
\begin{Value}
Returns an Rpath.sim object that can be supplied to the ecosim.run function.
\end{Value}
%
\begin{SeeAlso}\relax
Other Rpath.functions: \code{\LinkA{check.rpath.param}{check.rpath.param}};
\code{\LinkA{create.rpath.param}{create.rpath.param}}; \code{\LinkA{ecopath}{ecopath}};
\code{\LinkA{ecosim.plot}{ecosim.plot}}; \code{\LinkA{ecosim.run}{ecosim.run}};
\code{\LinkA{webplot}{webplot}}; \code{\LinkA{write.Rpath.sim}{write.Rpath.sim}};
\code{\LinkA{write.Rpath}{write.Rpath}}
\end{SeeAlso}
\inputencoding{utf8}
\HeaderA{ecosim.plot}{Plot routine for Ecosim runs}{ecosim.plot}
%
\begin{Description}\relax
Plots the relative biomass of each group from a run of ecosim.
\end{Description}
%
\begin{Usage}
\begin{verbatim}
ecosim.plot(Rpath.sim.obj)
\end{verbatim}
\end{Usage}
%
\begin{Arguments}
\begin{ldescription}
\item[\code{Rpath.sim.obj}] Rpath ecosim run created by the ecosim.run() function.
\end{ldescription}
\end{Arguments}
%
\begin{Value}
Creates a figure of relative biomass.
\end{Value}
%
\begin{SeeAlso}\relax
Other Rpath.functions: \code{\LinkA{check.rpath.param}{check.rpath.param}};
\code{\LinkA{create.rpath.param}{create.rpath.param}}; \code{\LinkA{ecopath}{ecopath}};
\code{\LinkA{ecosim.init}{ecosim.init}}; \code{\LinkA{ecosim.run}{ecosim.run}};
\code{\LinkA{webplot}{webplot}}; \code{\LinkA{write.Rpath.sim}{write.Rpath.sim}};
\code{\LinkA{write.Rpath}{write.Rpath}}
\end{SeeAlso}
\inputencoding{utf8}
\HeaderA{ecosim.run}{Ecosim modual of Rpath}{ecosim.run}
%
\begin{Description}\relax
Runs the ecosim modual.
\end{Description}
%
\begin{Usage}
\begin{verbatim}
ecosim.run(simpar, BYY = 0, EYY = 0, init_run = 0)
\end{verbatim}
\end{Usage}
%
\begin{Arguments}
\begin{ldescription}
\item[\code{simpar}] Rpath.sim object containing ecosim parameters (Generated by ecosim.init).

\item[\code{BYY,}] EYY Integer values for the beginning/end year.

\item[\code{init\_run}] Optional flag to declare if this is an inital run of the ecosim model.
\end{ldescription}
\end{Arguments}
%
\begin{Value}
Returns an Rpath.sim object.
\end{Value}
%
\begin{SeeAlso}\relax
Other Rpath.functions: \code{\LinkA{check.rpath.param}{check.rpath.param}};
\code{\LinkA{create.rpath.param}{create.rpath.param}}; \code{\LinkA{ecopath}{ecopath}};
\code{\LinkA{ecosim.init}{ecosim.init}}; \code{\LinkA{ecosim.plot}{ecosim.plot}};
\code{\LinkA{webplot}{webplot}}; \code{\LinkA{write.Rpath.sim}{write.Rpath.sim}};
\code{\LinkA{write.Rpath}{write.Rpath}}
\end{SeeAlso}
\inputencoding{utf8}
\HeaderA{webplot}{Plot routine for Ecopath food web}{webplot}
%
\begin{Description}\relax
Plots the food web associated with an Rpath object.
\end{Description}
%
\begin{Usage}
\begin{verbatim}
webplot(Rpath.obj, eco.name = attr(Rpath.obj, "eco.name"),
  line.col = "grey", highlight = NULL, highlight.col = c("black", "red",
  "orange"), labels = FALSE, label.pos = NULL, label.num = FALSE,
  label.cex = 1, fleets = FALSE, type.col = "black", box.order = NULL)
\end{verbatim}
\end{Usage}
%
\begin{Arguments}
\begin{ldescription}
\item[\code{Rpath.obj}] Rpath model created by the ecopath() function.

\item[\code{eco.name}] Optional name of the ecosystem.  Default is the eco.name attribute from the
rpath object.

\item[\code{line.col}] The color of the lines between nodes of the food web.

\item[\code{highlight}] Box number to highlight connections.

\item[\code{highlight.col}] Color of the connections to the highlighted group.

\item[\code{labels}] Logical whether or not to display group names.  If True and label.pos is Null, no
points will be ploted, just label names.

\item[\code{label.pos}] A position specifier for the labels.  Values of 1, 2, 3, 4, respectively
indicate positions below, to the left of, above, and to the right of the points. A null
value will cause the labels to be ploted without the points (Assuming that labels = TRUE).

\item[\code{label.num}] Logical value indication whether group numbers should be used for labels
instead of names.

\item[\code{label.cex}] The relative size of the labels within the plot.

\item[\code{fleets}] Logical value indicating whether or not to include fishing fleets in the food web.

\item[\code{type.col}] The color of the points cooresponding to the types of the group.  Can either be
of length 1 or 4.  Color order will be living, primary producers, detrital, and fleet groups.

\item[\code{box.order}] Vector of box numbers to change the default plot order.  Must include all box numbers

\item[\code{highlight}] Set to the group number to highlight the connections of that group.
\end{ldescription}
\end{Arguments}
%
\begin{Value}
Creates a figure of the food web.
\end{Value}
%
\begin{SeeAlso}\relax
Other Rpath.functions: \code{\LinkA{check.rpath.param}{check.rpath.param}};
\code{\LinkA{create.rpath.param}{create.rpath.param}}; \code{\LinkA{ecopath}{ecopath}};
\code{\LinkA{ecosim.init}{ecosim.init}}; \code{\LinkA{ecosim.plot}{ecosim.plot}};
\code{\LinkA{ecosim.run}{ecosim.run}}; \code{\LinkA{write.Rpath.sim}{write.Rpath.sim}};
\code{\LinkA{write.Rpath}{write.Rpath}}
\end{SeeAlso}
\inputencoding{utf8}
\HeaderA{write.Rpath}{Write function for Ecopath object}{write.Rpath}
%
\begin{Description}\relax
Outputs basic parameters or mortalities to a .csv file.
\end{Description}
%
\begin{Usage}
\begin{verbatim}
write.Rpath(x, file, morts = F, ...)
\end{verbatim}
\end{Usage}
%
\begin{Arguments}
\begin{ldescription}
\item[\code{x}] Rpath model created by the ecopath() function.

\item[\code{file}] file name for resultant .csv file.  Be sure to include ".csv".

\item[\code{morts}] Logical value whether to output basic parameters or mortalities.
\end{ldescription}
\end{Arguments}
%
\begin{Value}
Writes a .csv file with the basic parameters or mortalities from an Rpath object.
\end{Value}
%
\begin{SeeAlso}\relax
Other Rpath.functions: \code{\LinkA{check.rpath.param}{check.rpath.param}};
\code{\LinkA{create.rpath.param}{create.rpath.param}}; \code{\LinkA{ecopath}{ecopath}};
\code{\LinkA{ecosim.init}{ecosim.init}}; \code{\LinkA{ecosim.plot}{ecosim.plot}};
\code{\LinkA{ecosim.run}{ecosim.run}}; \code{\LinkA{webplot}{webplot}};
\code{\LinkA{write.Rpath.sim}{write.Rpath.sim}}
\end{SeeAlso}
\inputencoding{utf8}
\HeaderA{write.Rpath.sim}{Write function for Ecosim object}{write.Rpath.sim}
%
\begin{Description}\relax
Outputs start/end biomass and catch to a .csv file.
\end{Description}
%
\begin{Usage}
\begin{verbatim}
write.Rpath.sim(x, file, ...)
\end{verbatim}
\end{Usage}
%
\begin{Arguments}
\begin{ldescription}
\item[\code{x}] Rpath.sim object created by the ecosim.run() function.

\item[\code{file}] file name for resultant .csv file.  Be sure to include ".csv".
\end{ldescription}
\end{Arguments}
%
\begin{Value}
Writes a .csv file with the start and end biomass and catch per group
from an Rpath.sim object.
\end{Value}
%
\begin{SeeAlso}\relax
Other Rpath.functions: \code{\LinkA{check.rpath.param}{check.rpath.param}};
\code{\LinkA{create.rpath.param}{create.rpath.param}}; \code{\LinkA{ecopath}{ecopath}};
\code{\LinkA{ecosim.init}{ecosim.init}}; \code{\LinkA{ecosim.plot}{ecosim.plot}};
\code{\LinkA{ecosim.run}{ecosim.run}}; \code{\LinkA{webplot}{webplot}};
\code{\LinkA{write.Rpath}{write.Rpath}}
\end{SeeAlso}
\printindex{}
\end{document}
